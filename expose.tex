\documentclass{scrartcl}
\usepackage[utf8]{inputenc}
\usepackage{hyperref}

\DeclareUnicodeCharacter{202F}{'e}

\title{Schema Inference on Wikidata}
\subtitle{Exposé (Draft)}
\author{Lucas Werkmeister}

\begin{document}

\maketitle

\section{Goal}

The main goal of this thesis will be to automatically infer data schemata from Wikidata items:
given a set of exemplary items (e.\,g. some composers, some capitals, or some chemical elements),
determine which aspects all items have in common (mandatory constraints),
which aspects most of them have in common (non-mandatory constraints),
and which ones are particular to each individual item.

For example, one could infer that all composers have a statement “instance of: human”;
that all capitals have a statement “instance of: (some subclass of) city”;
and that most chemical elements have a “mass” statement, but with varying values.

\section{Details}

Data schemata can be expressed using the Shape Expressions Language (ShEx) \cite{Prud'hommeaux:2014:SER:2660517.2660523},
referring to the RDF output format of Wikidata.
The output of the inference process would therefore be a ShEx schema in one of the ShEx syntaxes, e.\,g. ShExC.
ShEx has previously been used to validate RDF data, e.\,g. in \cite{SOLBRIG201790}.

These schemata could be stored on Wikidata itself,
where they would be open to users amending and improving them.
Of course, the average Wikidata editor cannot be expected to be familiar with ShEx,
but since several community members are already investigating the use of ShEx on Wikidata,
a collaborative effort seems possible.

\section{Use}

A practical use of these schemata could be to check the compliance of other Wikidata items to these shapes,
in order to use them for quality control.
Running these checks periodically, and saving the results each time,
could yield an impression of the development of data quality on Wikidata over time.

The individual shapes constituting the schemata could also be used to streamline data entry into Wikidata:
the user selects the shape of the data they would like to enter,
and the user interface suggests statements based on the shape.

\section{Schema Inference vs. Ontology Learning}

The goal of this work is explicitly the inference of schemata of the data representation on Wikidata,
not the learning of an ontology on the data.
When an item is in violation of a schema,
for instance because an expected statement is missing,
this implies nothing more than the simple fact of the violation:
it is possible that the statement should be added
(in this case, an ontology might have inferred the statement),
but it is equally possible that something else should be changed about the item so that the missing statement is no longer expected,
or that the schema simply does not apply to the item and nothing should be done at all.

\bibliographystyle{plain}
\bibliography{expose}

\end{document}
